%%%%%%%%%%%%%%%%%%%%%%%%%%%%%%%%%%%%%%%%%%%%%%%%%%%%%%%%%%%%%%%%%%%%%%%%%%%
%
% Generic template for TFC/TFM/TFG/Tesis
%
% $Id: resumen.tex,v 1.8 2014/04/17 17:28:45 macias Exp $
%
% By:
%  + Javier Mac�as-Guarasa. 
%    Departamento de Electr�nica
%    Universidad de Alcal�
%  + Roberto Barra-Chicote. 
%    Departamento de Ingenier�a Electr�nica
%    Universidad Polit�cnica de Madrid   
% 
% Based on original sources by Roberto Barra, Manuel Oca�a, Jes�s Nuevo,
% Pedro Revenga, Fernando Herr�nz and Noelia Hern�ndez. Thanks a lot to
% all of them, and to the many anonymous contributors found (thanks to
% google) that provided help in setting all this up.
%
% See also the additionalContributors.txt file to check the name of
% additional contributors to this work.
%
% If you think you can add pieces of relevant/useful examples,
% improvements, please contact us at (macias@depeca.uah.es)
%
% Copyleft 2013
%
%%%%%%%%%%%%%%%%%%%%%%%%%%%%%%%%%%%%%%%%%%%%%%%%%%%%%%%%%%%%%%%%%%%%%%%%%%%

\chapter*{Resumen}
\label{cha:resumen}
\markboth{Resumen}{Resumen}

\addcontentsline{toc}{chapter}{Resumen}

Los sistemas de percepci�n en los veh�culos aut�nomos, son aquellos que permiten comprender lo que sucede en el entorno. Este trabajo se centra en el estudio de t�cnicas de percepci�n utilizando tecnolog�as LiDAR, tanto t�cnicas cl�sicas como t�cnicas basadas en Deep Learning. Con dicho estudio se presentan las implementaciones dise�adas para el veh�culo del proyecto Techs4AgeCar, ofreciendo la implementaci�n basada en Deep Learning una mejora en la detecci�n dentro de la capa de percepci�n utilizando unicamente LiDAR.\par
Junto con la implementaci�n basada en un solo sensor, se presenta un modelo se fusi�n sensorial entre c�mara y LiDAR basado en el sistema de detecci�n explicado en este TFG y el trabajo basado en c�mara realizado con un compa�ero del grupo de investigaci�n RobeSafe.\par
Por �ltimo para la evaluaci�n de sistemas de conducci�n aut�nomos sobre el simulador CARLA, se explica el proyecto AD DevKit, sobre el que se realiza la capa de percepci�n para la evaluaci�n de sistemas de detecci�n 2D y 3D en dicho simulador.

\textbf{Palabras clave:} \mybookpalabrasclave.

%%% Local Variables:
%%% TeX-master: "../book"
%%% End:


