%%%%%%%%%%%%%%%%%%%%%%%%%%%%%%%%%%%%%%%%%%%%%%%%%%%%%%%%%%%%%%%%%%%%%%%%%%%
%
% Generic template for TFC/TFM/TFG/Tesis
%
% $Id: conclusiones.tex,v 1.3 2014/01/08 22:56:04 macias Exp $
%
% By:
%  + Javier Mac�as-Guarasa. 
%    Departamento de Electr�nica
%    Universidad de Alcal�
%  + Roberto Barra-Chicote. 
%    Departamento de Ingenier�a Electr�nica
%    Universidad Polit�cnica de Madrid   
% 
% Based on original sources by Roberto Barra, Manuel Oca�a, Jes�s Nuevo,
% Pedro Revenga, Fernando Herr�nz and Noelia Hern�ndez. Thanks a lot to
% all of them, and to the many anonymous contributors found (thanks to
% google) that provided help in setting all this up.
%
% See also the additionalContributors.txt file to check the name of
% additional contributors to this work.
%
% If you think you can add pieces of relevant/useful examples,
% improvements, please contact us at (macias@depeca.uah.es)
%
% Copyleft 2013
%
%%%%%%%%%%%%%%%%%%%%%%%%%%%%%%%%%%%%%%%%%%%%%%%%%%%%%%%%%%%%%%%%%%%%%%%%%%%

\chapter{Conclusiones y l�neas futuras}
\label{cha:concl-y-line}

En este apartado se resumen las conclusiones obtenidas y se proponen
futuras l�neas de investigaci�n que se deriven del trabajo.

La estructura del cap�tulo es...


\section{Conclusiones}
\label{sec:conclusiones}

Para a�adir una referencia a un autor, se puede utilizar el paquete
\texttt{cite}. En el trabajo \cite{armani03}, se muestra un trabajo...

Y podemos usar de nuevo alg�n acr�nimo, como por ejemplo \ac{TDPSOLA}, o
uno ya referenciado como \ac{ANN}.


\section{L�neas futuras}
\label{sec:lineas-futuras}

Pues eso.


%%% Local Variables:
%%% TeX-master: "../book"
%%% End:


