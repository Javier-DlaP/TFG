\chapter{Conclusiones y trabajos futuros}
\label{cha:conlusiones_y_trabajos_futuros}

\begin{FraseCelebre}
  \begin{Frase}
    La verdadera felicidad radica en la finalizaci�n del trabajo utilizando tu propio cerebro y habilidades.
  \end{Frase}
  \begin{Fuente}
    Soichiro Honda
  \end{Fuente}
\end{FraseCelebre}

\noindent
Como finalizaci�n del Trabajo Fin de Grado se va a recapitular el trabajo realizado para ofrecer as� unas conclusiones y mostrar el futuro trabajo que se va a realizar como continuaci�n al estudio, implementaci�n y desarrollo de todo lo descrito en este TFG. 

\section{Conclusiones}
\label{sec:mconclusiones}

El objetivo principal de este trabajo, ha sido el estudio de t�cnicas cl�sicas de detecci�n con \acs{lidar} y los modelos \acs{sota} basados en Deep Learning que ofrecen la mayor precisi�n para la detecci�n de objetos 3D. Tras esto se implementar�a el mejor modelo encontrado, para que en el proyecto Techs4AgeCar del grupo RobeSafe se mejore el sistema de detecci�n con \acs{lidar} en el veh�culo T4AC.\par
Antes de comienzo del estudio y el desarrollo, ha sido necesario aprender a utilizar las herramientas como ROS, Docker o CARLA que son utilizadas en el proyecto, para as� poder comprender el funcionamiento del sistema de conducci�n completo actual y ayudar a mejorarlo.\par
Se han estudiado m�ltiples t�cnicas como la voxelizaci�n, \acs{ransac} 3D o KD-tree para la aplicaci�n en modelos cl�sicos. Con todo esto se ha construido un modelo de detecci�n 3D basado en \acs{lidar} a partir de unicamente t�cnicas cl�sicas, programado en C++ como un nodo de ROS para as� poder utilizar dicho modelo en CARLA.\par
Tras esto se han analizado diferentes datasets como Kitti, nuScenes y Waymo, adem�s de los modelos \acs{sota} que m�s adelante han sido evaluados. Con dicho estudio se ha implementado en el proyecto Techs4AgeCar un sistema de detecci�n 3D con \acs{lidar} basado en el modelo CBGS con un backbone cambiado al de PointPillars.\par
Con esto se habr�a terminado la base del TFG pero no se ha dejado ah�, en conjunto con Carlos G�mez Hu�lamo y Miguel Antunes Garc�a se ha dise�ado un sistema de fusi�n sensorial que mejore las detecciones 3D haciendo uso de las detecci�n provenientes de la c�mara y el \acs{lidar}. Y eso no ha sido todo, ya que al finalizar esto se ha desarrollado el comienzo de lo que ser� la capa de percepci�n del \acs{ad_devkit} para permitir la evaluaci�n de los sistemas de percepci�n en CARLA.\par
Este trabajo ha conseguido obtener todo lo propuesto en t�rminos te�ricos y pr�cticos, adem�s de haber continuado trabajando con lo aprendido en este TFG para conseguir as� un trabajo mucho m�s completo al realizar nuevas tareas, obteniendo durante todo el trabajo un mayor conocimiento de los sistemas de percepci�n y esperando que el trabajo aporte un valor al grupo RobeSafe.

\section{Trabajos futuros}
\label{sec:trabajos_futuros}


