\chapter{Sistemas de percepci�n con LiDAR basados en Deep Learning}
\label{cha:sistemas_de_percepcion_con_lidar_basados_en_deep_learning}

%\begin{FraseCelebre}
%  \begin{Frase}
%    Desocupado lector, sin juramento me podr�s creer que quisiera que este
%    libro [...] fuera el m�s hermoso, el m�s gallardo y m�s discreto que
%    pudiera imaginarse\footnote{Tomado de ejemplos del proyecto \texis{}.}.
%  \end{Frase}
%  \begin{Fuente}
%    Miguel de Cervantes, Don Quijote de la Mancha
%  \end{Fuente}
%\end{FraseCelebre}


\section{Principales datasets}
\label{sec:principales_datasets}



\subsection{Kitti}
\label{sec:kitti}



\subsection{Waymo}
\label{sec:waymo}



\subsection{nuScenes}
\label{sec:nuscenes}



\subsubsection{nuScenes-devkit}
\label{sec:nuscenes_devkit}



\subsection{Comparativa entre los diferentes datasets}
\label{sec:comparativa_entre_los_diferentes_datasets}



\section{Estado del arte en detecci�n utilizando LiDAR}
\label{sec:estado_del_arte_en_deteccion_utilizando_lidar}



\subsection{PointPillars}
\label{sec:pointpillars}



\subsection{SECOND}
\label{sec:second}



\subsection{PointRCNN}
\label{sec:pointrcnn}



\subsection{PV-RCNN}
\label{sec:pv_rcnn}



\subsection{CBGS}
\label{sec:cbgs}



\section{OpenPCDet}
\label{sec:openpcdet}


