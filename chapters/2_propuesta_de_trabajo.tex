\chapter{Propuesta de trabajo}
\label{cha:propuesta_de_trabajo}

\begin{FraseCelebre}
  \begin{Frase}
    La educaci�n cient�fica de los j�venes es al menos tan importante, quiz� incluso m�s, que la propia investigaci�n.
  \end{Frase}
  \begin{Fuente}
    Glenn Theodore Seaborg
  \end{Fuente}
\end{FraseCelebre}

\noindent
Este trabajo se encuentra dentro del proyecto Tech4AgeCars perteneciente al grupo RobeSafe. En este se pretende realizar un estudio de diferentes t�cnicas de detecci�n utilizando \acs{lidar}, tanto con m�todos cl�sicos como basados en Deep Learning para el an�lisis en datasets reales, en el simulador CARLA y en el coche del grupo RobeSafe.\par
En la arquitectura del proyecto se tiene implementado un procesamiento basado en PointPillars para una ejecuci�n en tiempo real sobre una plataforma NVIDIA Jetson AGX Xavier \cite{tfm_del_egido}, por lo que se pretende remplazar esta subtarea de detecci�n utilizando \acs{lidar}, dentro de la capa de percepci�n.\par
Con la implementaci�n de la detecci�n realizada, se pretende realizar una fusi�n sensorial entre c�mara y \acs{lidar} junto con un compa�ero del grupo RobeSafe que se encuentra realizando un TFG de detecci�n 3D utilizando c�mara \cite{tfg_miguel}.\par
Tras una fusi�n y sin la posibilidad de analizar el sistema de percepci�n a crear, se trabajar� por �ltimo en un proyecto junto con el \ac{kit} llamado \ac{ad_devkit} que tratar� de analizar un \acl{ads}, en concreto se trabajar� en el apartado de percepci�n del kit de desarrollo, todo ello esperando que este proyecto no solo sea �til para los compa�eros del proyecto sino para cualquier desarrollador que utilice el simulador CARLA. 
